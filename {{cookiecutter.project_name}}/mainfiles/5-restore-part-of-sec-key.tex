%!TEX root = ../{{cookiecutter.project_name}}.tex

\section{Восстановление части ключа}

\begin{frame}
\frametitle{Восстановление части ключа}
\begin{block}{Определение}
 $\widehat{\mathcal A}_u(RM(r,m))$~--- это множество перестановок вида $\nabla \aurm{\gamma}$, где $\aurm{\gamma}$~--- это перестановка из группы расширенных автоморфизмов $\mathcal A_u(RM(r,m))$, а $\nabla$~--- это произвольная перестановка блоков матрицы $(\matr{H}_1\matr{R}\|\ldots\|\matr{H}_u\matr{R})$.
\end{block}
\begin{block}{Утвеждение}
Справедливо равенство
\begin{equation*}
\bigcap_{\matr{H}_1,\ldots,\matr{H}_u\in GL(k,2)}\mathcal G(\matr{H}_1,\ldots,\matr{H}_u)=\widehat{\mathcal A}_u(RM(r,m)),
\end{equation*}
\end{block}
\end{frame}


\begin{frame}
\frametitle{Восстановление части ключа}
\begin{block}{Теорема 4}
Пусть перестановка $\Gamma=\Gamma_{I\leftrightarrow J}\gamma[1]\sigma[2]$ принадлежит множеству $\mathcal G(\matr{E},\matr{H})$.  Тогда используя эту перестановку, можно построить $p_{\overline{I}}+p_{\overline{J}}$ линейно независимых уравнений относительно $n$ неизвестных $\matr{H}R_{1}$, $\matr{H}R_{2}$, $\ldots$, $\matr{H}R_{n}$. Здесь $R_i$ столбец с номером $i$ порождающей матрицы кода Рида--Маллера $RM(r,m)$.
\end{block}
\end{frame}
