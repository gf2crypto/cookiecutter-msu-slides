%!TEX root = {{cookiecutter.project_name}}.tex

% Добавьте ссылку на файлы с текстом работы
% Можно использовать команды:
%   \input или \include
% Пример:
%    \input{mainfiles/1-section} или \include{mainfiles/2-section}
% Команда \input позволяет включить текст файла без дополнительной обработки
% Команда \include при включении файла добавляет до него и после него команду
% перехода на новую страницу. Кроме того, она позволяет компилировать каждый файл
% в отдельности, что ускоряет сборку проекта.
% ВАЖНО: команда \include не поддерживает включение файлов, в которых уже содержится команда \include,
% т.е. не возможен рекурсивный вызов \include
\newcommand*{\Source}{
    %!TEX root = ../slides.tex

\section{Криптосистема Мак-Элиса}

\begin{frame}
  \frametitle{Криптосистемы Мак-Элиса}
  %\framesubtitle{Subtitles are optional.}
  % - A title should summarize the slide in an understandable fashion
  %   for anyone how does not follow everything on the slide itself.

  \begin{itemize}
  \item
    Криптосистема с открытым ключом. Предложена в 1978 году Р.Дж.~Мак-Элисом.
  \pause
  \item
    Стойкость основана на трудности некоторых задачах теории кодов, исправляющих ошибки.
  \pause
  \item
        Р.Дж.~Мак-Элис предложил использовать  семейство кодов
        Гоппы с параметрами $[1024,524,101]$;
        \pause
  \item
        Г. Нидеррайтер в 1986 году предложил использовать обобщённые коды
        Рида--Соломона. В 1992 году В.М.~Сидельников и
        С.О.~Шестаков взломали криптосистему Мак-Элиса на основе
        этих кодов.
        \pause
  \item
        В.М.~Сидельников в 1994 году предложил использовать двоичные коды Рида--Маллера $RM(r,m)$.
  \end{itemize}
\end{frame}

    %!TEX root = ../slides.tex

\section{Криптосистема Мак-Элиса--Сидельникова }

\subsection[Устройство]{Устройство криптосистемы Мак-Элиса--Сидельникова}

\begin{frame}
  \frametitle{Общие сведения}
\begin{itemize}
  \item Предложена В.М.~Сидельниковым в 1994 году как альтернатива криптосистеме Мак-Элиса.
  \pause
  \item Оригинальная криптосистема строится на основе двоичных кодов Рида--Маллера $RM(r,m)$.
  \end{itemize}

\end{frame}

\begin{frame}
  \frametitle{Секретный и открытый ключ.}
\begin{itemize}
  \item Параметры:
  \begin{itemize}
    \item
      $r$~--- натуральное число;
    \item
      $m$~--- натуральное число, $m\geqslant r$;
   \item $R$~--- порождающая матрица кода Рида--Маллера $RM(r,m)$;
   \item $u$~--- натуральное число.
    \end{itemize}
    \pause
  \item Секретный ключ:
    \begin{itemize}
    \item
      $H_1,\ldots,H_u$~--- невырожденные $(k\times k)$-матрицы над полем $GF(2)$.
    \item
      $\Gamma$~--- перестановочная $(un\times un)$-матрица, $\Gamma\in S_{u\cdot n}$.
    \end{itemize}
    \pause
  \item
    Открытый ключ~--- матрица
    $$G'=(H_1R\|H_2R\|\ldots\|H_uR)\Gamma.$$
  \end{itemize}

\end{frame}


\subsection{Пространство ключей}

\begin{frame}
  \frametitle{Основные определения.}
  \begin{itemize}
    \item
        Секретные ключи $(H'_1,\dots,H'_u,\Gamma')$ и $(H''_1,\ldots,H''_u,\Gamma'')$
        называются \alert{эквивалентными}, если
        $$
        (H'_1R\|\ldots\|H'_uR)\Gamma'=(H''_1R\|\ldots\|H''_uR)\Gamma''
        $$
    \item
        $[(H_1,\ldots,H_u,\Gamma)]$~--- класс эквивалентности с
        представителем $(H_1,\ldots,H_u,\Gamma)$.
        \pause
    \item
        Введём множество $\mathcal G(H_1,\ldots,H_u)$:
        \begin{eqnarray*}
            \mathcal G(H_1,\ldots,H_u)=\{\Gamma\in S_{un}|
            \exists H'_1,\ldots, H'_u\text{ такие, что}\\
            (H_1R\|\ldots\|H_uR)\Gamma=(H'_1R\|\ldots\|H'_uR)\}
        \end{eqnarray*}
  \end{itemize}

\end{frame}

\begin{frame}
  \frametitle{Оценка мощности множества открытых ключей}
\begin{block}{Теорема 1}
Справедливы неравенства для числа $|\mathcal E|$ открытых ключей криптосистемы
Мак-Элиса--Сидельникова с $u>1$ блоками на основе кодов Рида--Маллера $RM(r,m)$
$$
\frac{(u\cdot n)!h_k}{(u!)^n|Aut(RM(r,m))|}\leqslant |\mathcal E|<\frac{(u\cdot
n)!(h_k)^u}{u!|Aut(RM(r,m))|^u}.
$$
Здесь
\begin{itemize}
\item $n$~--- длина кода Рида--Маллера $RM(r,m)$,
\item $h_k$~--- число обратимых $(k\times k)$-матриц над полем $GF(2)$,
\item $Aut(RM(r,m))$~--- группа автоморфизмов кода $RM(r,m)$.
\end{itemize}
\end{block}
Оценка сверху принадлежит Г.А. Карпунину (2004).
\end{frame}

    %!TEX root = ../slides.tex

\section[Описание открытых ключей]{Описание множества открытых ключей}

\subsection[Произвольное число блоков]{Случай произвольного числа блоков}

\begin{frame}
  \frametitle{Множество эквивалентных ключей.}
\begin{block}{Теорема 2a}
Пусть
\begin{itemize}
\item матрицы $D_1,D_2,\ldots, D_u$ задают автоморфизмы $\sigma_1,\sigma_2,\ldots,\sigma_u$ кода Рида--Маллера $RM(r,m)$;
\item $\sigma_j[i]$~--- расширенный автоморфизм, соответствующий $\sigma_j$,
\item $H$~--- любая невырожденная матрица.
\end{itemize}
Тогда класс эквивалентности $[(HD_1,HD_2,\ldots,HD_u,\Gamma)]$ состоит из кортежей вида
\begin{eqnarray*}
(HA_1,HA_2,\ldots,HA_u,\gamma^{-1}_1[1]\cdot\gamma^{-1}_2[2]\ldots
\gamma^{-1}_u[u]\Gamma'
\cdot\sigma_1[1]\cdot\sigma_2[2]\ldots\sigma_u[u]\Gamma),
\end{eqnarray*}
здесь для $\Gamma'$ выполнено $(R\|R\|\ldots\|R)\Gamma'=(R\|R\|\ldots\|R).$
\end{block}
\end{frame}

\subsection[Два блока]{Случай двух блоков}

\begin{frame}
\frametitle{Специальный тип матриц.}
Рассмотрим матрицу $T^I_{\widetilde{A}}$, $I=\{i_1,i_2,\ldots,i_p\}$ вида
\begin{equation*}
\left(\begin{tabular}{ccccccccccc}
&               &         &       &$i_1$        &       &$i_2$       &      &$i_p$       &      &\\
&               &         &       &$\downarrow$ &       &$\downarrow$&      &$\downarrow$&       &\\
&1              &0        &$\ldots$ &0          &$\ldots$ &0         &$\ldots$&0         &$\ldots$&0\\
&0              &1        &$\ldots$ &0          &$\ldots$ &0         &$\ldots$&0         &$\ldots$&0\\
&$\vdots$         &$\vdots$   &$\ldots$ &$\vdots$     &$\ldots$ &$\vdots$    &$\ldots$&$\vdots$    &$\ldots$&$\vdots$     \\
$i_1 \rightarrow$ &$\alpha^{i_1}_1$ &$\alpha^{i_1}_2$ &$\ldots$ &$\alpha^{i_1}_{i_1}$&$\ldots$&$\alpha^{i_1}_{i_2}$&$\ldots$ &$\alpha^{i_1}_{i_p}$&$\ldots$ &$\alpha^{i_1}_{k}$ \\
               &$\vdots$   &$\vdots$   &$\ldots$ &$\vdots$&\ldots&$\vdots$     &$\ldots$ &$\vdots$&$\ldots$ &$\vdots$     \\
$i_2\rightarrow$ &$\alpha^{i_2}_1$ &$\alpha^{i_2}_2$ &$\ldots$ &$\alpha^{i_2}_{i_1}$&$\ldots$ &$\alpha^{i_2}_{i_2}$&$\ldots$ &$\alpha^{i_2}_{i_p}$        &$\ldots$ &$\alpha^{i_2}_{k}$ \\
               &$\vdots$   &$\vdots$   &$\ldots$ &$\vdots$&$\ldots$ &$\vdots$&$\ldots$ &$\vdots$      &$\ldots$ &$\vdots$     \\
$i_p\rightarrow$ &$\alpha^{i_p}_1$ &$\alpha^{i_p}_2$ &$\ldots$ &$\alpha^{i_p}_{i_1}$&$\ldots$ &$\alpha^{i_p}_{i_2}$  &$\ldots$ &$\alpha^{i_p}_{i_p}$      &$\ldots$ &$\alpha^{i_p}_{k}$ \\
               &$\vdots$   &$\vdots$   &$\ldots$ &$\vdots$ &$\ldots$ &$\vdots$&$\ldots$ &$\vdots$        &$\ldots$ &$\vdots$     \\
               &0        &0        &$\ldots$ &0 &$\ldots$ &0     &$\ldots$ &0     &$\ldots$ &1          \\
\end{tabular}\right),
\end{equation*}
\end{frame}

\begin{frame}
\frametitle{Первый случай. $|I|=1$,  $I=\{i\}$.}
\begin{block}{Теорема 2b}
Класс эквивалентности $[(\matr{H},\matr{H}\matr{T}^{i}_{\widetilde{\vect{\alpha}}},\Gamma)]$ состоит из кортежей вида
\begin{equation*}(\matr{H}\matr{T}^{i}_{\widetilde{\vect{\beta}}}\matr{D}_1,\matr{H}\matr{T}^{i}_{\widetilde{\vect{\gamma}}}\matr{D}_2,\aurm{\sigma_L}^{-1}[1]\aurm{\sigma}_R^{-1}[2]\Gamma'^{-1}\Gamma).\end{equation*}
Здесь $\aurm{\sigma}_L,\aurm{\sigma}_R$~--- автоморфизмы кода Рида--Маллера $RM(r,m)$, соответствующие матрицам $\matr{D_1}$ и $\matr{D_2}$, а для перестановки $\Gamma'$ выполняются два условия
\begin{itemize}
\item[1)] Если $\matr{R}'$~--- $(k-1)\times n$-матрица, получающаяся удалением строки с номером $i$ из матрицы $\matr{R}$, то
$(\matr{R}'\|\matr{R}')\Gamma'=(\matr{R}'\|\matr{R}');$
\item[2)] Если $\vectrow{r_i}$~--- строка матрицы $\matr{R}$ с номером $i$, то
\begin{equation*}(\vectrow{r_i}\|\widetilde{\vect{\alpha}} \matr{R})\Gamma'=(\widetilde{\vect{\beta}} \matr{R}\|\widetilde{\vect{\gamma}} \matr{R})\in RM(r,m)\times RM(r,m).\end{equation*}
\end{itemize}
\end{block}
\end{frame}


\begin{frame}
\frametitle{Третий случай. $|I|>1$}
\begin{block}{Теорема 2c}
Пусть $\Gamma^{-1}_g$~--- перестановка из $\mathcal G(\matr{E},\matr{T}^{I}_{\widetilde{A}})$, представимая в виде $\Gamma'\sigma_L[1]\sigma_R[2]$, где $\Gamma'$ такая перестановка, что $(\matr{R}'\|\matr{R}')\Gamma'=(\matr{R}'\|\matr{R}').$ Тогда класс эквивалентности $[(\matr{H},\matr{H}\matr{T}^I_{\widetilde{A}}, \Gamma)]$, содержит кортежи вида
\begin{equation*}(\matr{H}\matr{T}^{I}_{\widetilde{B}}\matr{D}_1,\matr{H}\matr{T}^{I}_{\widetilde{C}}\matr{D}_2,\aurm{\sigma_L}^{-1}[1]\aurm{\sigma}_R^{-1}[2]\Gamma'^{-1}\Gamma).\end{equation*}
Здесь $\aurm{\sigma}_L,\aurm{\sigma}_R$~--- автоморфизмы кода Рида--Маллера $RM(r,m)$, соответствующие матрицам $\matr{D_1}$ и $\matr{D_2}$, $\widetilde{B}=\{\widetilde{\beta}^{i_1},\widetilde{\beta}^{i_2}, \ldots, \widetilde{\beta}^{i_p}\}$, $\widetilde{C}=\{\widetilde{\gamma}^{i_1},\widetilde{\gamma}^{i_2}, \ldots, \widetilde{\gamma}^{i_p}\}$, а для перестановки $\Gamma'$ выполняется условие
\[(\vectrow{r_i}\|\widetilde{\vect{\alpha}}^{i} \matr{R})\Gamma'=(\widetilde{\vect{\beta}}^{i} \matr{R}\|\widetilde{\vect{\gamma}}^{i} \matr{R})\in RM(r,m)\times RM(r,m) \text{ для любого } i\in I.\]
\end{block}
\end{frame}

    %!TEX root = ../slides.tex

\section[Об эквивалентности]{О полиномиальной эквивалентности криптосистем Мак-Элиса и Мак-Элиса--Сидельникова с ограничениями на ключевое пространство.}

\begin{frame}
\frametitle{Задача mcRMi}
\begin{block}{Вход}
Число $m$ большее $2r$ и  $1\leqslant i\leqslant k$, матрица $\matr{G}=\matr{H}'\cdot \matr{R}'\cdot \gamma'$, где $\matr{H}'$~--- невырожденная двоичная $(k-1)\times (k-1)$-матрица, $\matr{R}'$~--- $((k-1)\times n)$-матрица, получающаяся из порождающей матрицы $\matr{R}$ кода Рида--Маллера $RM(r,m)$ выкидыванием строки с номером $i$ и $\gamma'$~--- перестановочная $(n\times n)$-матрица.
\end{block}
\begin{block}{Найти}
 Невырожденную матрицу $\matr{M}'$ размера $(k-1)\times (k-1)$ и перестановочную $(n\times n)$-матрицу $\sigma'$, для которых  найдётся невырожденная $((k-1)\times (k-1))$-матрица $\matr{L}'$, что  \[\matr{M}'\cdot \matr{G}\cdot \sigma'=\matr{L}'\cdot \matr{R}'.\]
\end{block}
\end{frame}


\begin{frame}
\frametitle{Задача mcSRM}
\begin{block}{Вход}
Матрица $\matr{G}=(\matr{H}_1\cdot \matr{R}\|\matr{H}_2\cdot \matr{R})\cdot \Delta$, где $\matr{H}_1$ и $\matr{H}_2$~--- невырожденные двоичные $(k\times k)$-матрицы, принадлежащие классу эквивалентности $[(\matr{H},\matr{H}\matr{T}^i_{\widetilde\alpha},\Gamma)]$  и $\Delta$~--- перестановочная $(2n\times 2n)$-матрица.
\end{block}
\begin{block}{Найти}
Невырожденные матрицы $\matr{H}'_1$ и $\matr{H}'_2$ размера $(k\times k)$ и перестановочную $(2n\times 2n)$-матрицу $\Delta'$ такие, что \[\matr{G}\cdot \Delta'=(\matr{H}'_1\matr{R}\|\matr{H}'_2\matr{R}).\]
\end{block}
\end{frame}

\begin{frame}
\frametitle{Эквивалентность}
\begin{block}{Теорема 3}
Пусть существует алгоритм, который решает задачу mcRMi за полиномиальное время.
Тогда существует алгоритм, который решает задачу mcSRM за полиномиальное время.
\end{block}
\end{frame}

    %!TEX root = ../slides.tex

\section{Восстановление части ключа}

\begin{frame}
\frametitle{Восстановление части ключа}
\begin{block}{Определение}
 $\widehat{\mathcal A}_u(RM(r,m))$~--- это множество перестановок вида $\nabla \aurm{\gamma}$, где $\aurm{\gamma}$~--- это перестановка из группы расширенных автоморфизмов $\mathcal A_u(RM(r,m))$, а $\nabla$~--- это произвольная перестановка блоков матрицы $(\matr{H}_1\matr{R}\|\ldots\|\matr{H}_u\matr{R})$.
\end{block}
\begin{block}{Утвеждение}
Справедливо равенство
\begin{equation*}
\bigcap_{\matr{H}_1,\ldots,\matr{H}_u\in GL(k,2)}\mathcal G(\matr{H}_1,\ldots,\matr{H}_u)=\widehat{\mathcal A}_u(RM(r,m)),
\end{equation*}
\end{block}
\end{frame}


\begin{frame}
\frametitle{Восстановление части ключа}
\begin{block}{Теорема 4}
Пусть перестановка $\Gamma=\Gamma_{I\leftrightarrow J}\gamma[1]\sigma[2]$ принадлежит множеству $\mathcal G(\matr{E},\matr{H})$.  Тогда используя эту перестановку, можно построить $p_{\overline{I}}+p_{\overline{J}}$ линейно независимых уравнений относительно $n$ неизвестных $\matr{H}R_{1}$, $\matr{H}R_{2}$, $\ldots$, $\matr{H}R_{n}$. Здесь $R_i$ столбец с номером $i$ порождающей матрицы кода Рида--Маллера $RM(r,m)$.
\end{block}
\end{frame}

    %!TEX root = ../slides.tex

\section{Заключение}

\begin{frame}
\frametitle{Основные результаты диссертации}
\begin{itemize}
\item[1)] Получена нижняя оценка мощности множества открытых ключей криптосистемы Мак-Элиса--Сидельникова~--- \alert{Теорема~1};
\item[2)] Описан ряд классов эквивалентности секретных ключей криптосистемы Мак-Элиса--Сидельникова~--- \alert{Теорема~2a}, \alert{Теорема~2b}, \alert{Теорема~2c};
\item[3)] Доказана полиномиальная эквивалентность задачи восстановления секретного ключа по открытому оригинальной криптосистемы Мак-Элиса и аналогичной задачи для криптосистемы Мак-Элиса--Сидельникова с ограничениями на ключевое пространство~--- \alert{Теорема~3};
\item[4)] Предложен метод восстановления части секретного ключа криптосистемы Мак-Элиса--Сидельникова, использующий знание структуры класса эквивалентности, в который попадает секретный ключ~--- \alert{Теорема~4}.
\end{itemize}
\end{frame}

    %!TEX root = ../slides.tex

\section{Список публикаций}

\begin{frame}
\frametitle{Список публикаций (3 из 7), в журналах  ВАК --- 2.}
\nocite{*}
\begin{alertblock}{Из списка ВАК}
\printbibliography[keyword=VAK, title=Из списка ВАК]
\end{alertblock}
\begin{exampleblock}{Другие}
\printbibliography[keyword=NoVAK, title=Из списка ВАК]
\end{exampleblock}
\end{frame}

}


% Информация о годе выполнения работы
\newcommand{\Date}{%
    % 16 июня 2010 г.%
    \today%     % Текущий день
}

% Укажите тип работы
% Например:
%     Выпускная квалификационная работа,
%     Магистерская диссертация,
%     Курсовая работа, реферат и т.п.
\newcommand{\WorkType}{%
    % Выпускная квалификационная работа%
    % Магистерская диссертация%
    % Курсовая работа%
    % Реферат%
    Дипломная работа%
}

% Название работы
%%%%%%%%%%% ВНИМАНИЕ! %%%%%%%%%%%%%%%%
% В МГУ ОНО ДОЛЖНО В ТОЧНОСТИ
% СООТВЕТСТВОВАТЬ ВЫПИСКЕ ИЗ ПРИКАЗА
% УТОЧНИТЕ НАЗВАНИЕ В УЧЕБНОЙ ЧАСТИ
\newcommand{\Title}{%
    Ключевое пространство криптосистемы Мак-Элиса--Сидельникова%
}


% Имя автора работы
\newcommand{\Author}{%
    Чижов Иван Владимирович%
}

% Информация о научном руководителе
%% Фамилия Имя Отчество%
\newcommand{\SciAdvisor}{%
    Карпунин Григорий Анатольевич%
}
%% В формате: И.~О.~Фамилия%
\newcommand{\SciAdvisorShort}{%
    Г.~А.~Карпунин%
}
%% должность научного руководителя
\newcommand{\Position}{%
    % профессор%
    доцент%
    % старший преподаватель%
    % преподаватель%
    % ассистент%
    % ведущий научный сотрудник%
    % старший научный сотрудник%
    % научный сотрудник%
    % младший научный сотрудник%
}
%% учёная степень научного руководителя
\newcommand{\AcademicDegree}{%
    % д.ф.-м.н.%
    % д.т.н.%
    к.ф.-м.н.%
    % к.т.н.%
    % без степени%
}

% Информация об организации, в которой выполнена работа
%% Город
\newcommand{\Place}{%
    Москва%
}
%% Университет
\newcommand{\Univer}{%
    Московский государственный университет имени М.~В.~Ломоносова%
}
\newcommand*{\UniverAbbr}{%
    МГУ%
}
%% Факультет
\newcommand{\Faculty}{%
    Факультет вычислительной математики и кибернетики%
}
%% Кафедра    
\newcommand{\Department}{%
    Кафедра информационной безопасности%
}     

%%%% Переключите формат экрана
\newcommand{\Aspect}{%
    % 43%
    % 1610%
    169%
}
