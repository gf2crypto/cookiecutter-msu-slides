%!TEX root = ../slides.tex

\section{Криптосистема Мак-Элиса--Сидельникова }

\subsection[Устройство]{Устройство криптосистемы Мак-Элиса--Сидельникова}

\begin{frame}
  \frametitle{Общие сведения}
\begin{itemize}
  \item Предложена В.М.~Сидельниковым в 1994 году как альтернатива криптосистеме Мак-Элиса.
  \pause
  \item Оригинальная криптосистема строится на основе двоичных кодов Рида--Маллера $RM(r,m)$.
  \end{itemize}

\end{frame}

\begin{frame}
  \frametitle{Секретный и открытый ключ.}
\begin{itemize}
  \item Параметры:
  \begin{itemize}
    \item
      $r$~--- натуральное число;
    \item
      $m$~--- натуральное число, $m\geqslant r$;
   \item $R$~--- порождающая матрица кода Рида--Маллера $RM(r,m)$;
   \item $u$~--- натуральное число.
    \end{itemize}
    \pause
  \item Секретный ключ:
    \begin{itemize}
    \item
      $H_1,\ldots,H_u$~--- невырожденные $(k\times k)$-матрицы над полем $GF(2)$.
    \item
      $\Gamma$~--- перестановочная $(un\times un)$-матрица, $\Gamma\in S_{u\cdot n}$.
    \end{itemize}
    \pause
  \item
    Открытый ключ~--- матрица
    $$G'=(H_1R\|H_2R\|\ldots\|H_uR)\Gamma.$$
  \end{itemize}

\end{frame}


\subsection{Пространство ключей}

\begin{frame}
  \frametitle{Основные определения.}
  \begin{itemize}
    \item
        Секретные ключи $(H'_1,\dots,H'_u,\Gamma')$ и $(H''_1,\ldots,H''_u,\Gamma'')$
        называются \alert{эквивалентными}, если
        $$
        (H'_1R\|\ldots\|H'_uR)\Gamma'=(H''_1R\|\ldots\|H''_uR)\Gamma''
        $$
    \item
        $[(H_1,\ldots,H_u,\Gamma)]$~--- класс эквивалентности с
        представителем $(H_1,\ldots,H_u,\Gamma)$.
        \pause
    \item
        Введём множество $\mathcal G(H_1,\ldots,H_u)$:
        \begin{eqnarray*}
            \mathcal G(H_1,\ldots,H_u)=\{\Gamma\in S_{un}|
            \exists H'_1,\ldots, H'_u\text{ такие, что}\\
            (H_1R\|\ldots\|H_uR)\Gamma=(H'_1R\|\ldots\|H'_uR)\}
        \end{eqnarray*}
  \end{itemize}

\end{frame}

\begin{frame}
  \frametitle{Оценка мощности множества открытых ключей}
\begin{block}{Теорема 1}
Справедливы неравенства для числа $|\mathcal E|$ открытых ключей криптосистемы
Мак-Элиса--Сидельникова с $u>1$ блоками на основе кодов Рида--Маллера $RM(r,m)$
$$
\frac{(u\cdot n)!h_k}{(u!)^n|Aut(RM(r,m))|}\leqslant |\mathcal E|<\frac{(u\cdot
n)!(h_k)^u}{u!|Aut(RM(r,m))|^u}.
$$
Здесь
\begin{itemize}
\item $n$~--- длина кода Рида--Маллера $RM(r,m)$,
\item $h_k$~--- число обратимых $(k\times k)$-матриц над полем $GF(2)$,
\item $Aut(RM(r,m))$~--- группа автоморфизмов кода $RM(r,m)$.
\end{itemize}
\end{block}
Оценка сверху принадлежит Г.А. Карпунину (2004).
\end{frame}
